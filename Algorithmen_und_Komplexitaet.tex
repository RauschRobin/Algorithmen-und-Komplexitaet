\documentclass[12pt,a4paper]{article}

\usepackage[utf8]{inputenc}
\usepackage[ngerman]{babel}
\usepackage[T1]{fontenc}
\usepackage{amsmath}
\usepackage{amsfonts}
\usepackage{amssymb}
\usepackage{graphicx}
\usepackage[left=2cm,right=2cm,top=2cm,bottom=2cm]{geometry}
\usepackage{multicol}
\usepackage{booktabs}
\usepackage[hidelinks]{hyperref}
\usepackage{tikz}
\usepackage{pgfplots}
\usepackage{blindtext}
\usepackage{array}
\usepackage{multirow}
\usepackage{bigdelim}
\usepackage{colortbl}
\usepackage{fancyhdr} 
\usepackage{tabularx}
\usepackage{xcolor}
\usepackage{color}
\usetikzlibrary{decorations.text}
\usetikzlibrary{tikzmark}
\pagestyle{fancy} 
	\fancyhf{} 
	\fancyhead[L]{\includegraphics[scale=0.05]{Bilder/dhbw.png}} 
	\fancyhead[C]{\slshape Algorithmen und Komplexitaet} 
	\fancyhead[R]{\slshape LaTeX Version}

\usepackage{helvet}
\renewcommand{\familydefault}{\sfdefault}

\author{\slshape Robin Rausch, Florian Maslowski}
\title{Algorithmen und Komplexität}
\date{\slshape \today}
\begin{document}
\maketitle
\tableofcontents
\newpage
\section{Komplexitaet}
Der Begriff Komplexität beschreibt...
\subsection{O-Notation}

\subsection{Logarithmen}

\subsection{Dynamisches Programmieren}

\subsection{Rekurrenzen}

\subsection{Divide \& Conquer}

\section{Einfache Sortierverfahren}
\subsection*{Selectionsort}
\begin{minipage}[t]{0.7\textwidth}
	\begin{enumerate}
		\item Finde kleinstes Element in Folge($a_0, ...a_{k-1}$)
		\item Vertausche $a_{min}$ mit $a_0$
		\item finde kleinstes Element in Folge($a_1, ...a_{k-1}$)
		\item Vertausche $a_{min}$ mit $a_1$
		\item ...
	\end{enumerate}
\end{minipage}
\begin{minipage}[t]{0.3\textwidth}
	Beispiel:
	\begin{center}
		\begin{tikzpicture}[scale=0.6]
			\foreach \y in {0,...,4}
				{\draw[black,very thick] (-2.5, 0.5 - \y * 2) rectangle (2.5,-0.5 - \y * 2);
				\draw[black,very thick] (-1.5, 0.5 - \y * 2) -- (-1.5,-0.5 - \y * 2);
				\draw[black,very thick] (-0.5, 0.5 - \y * 2) -- (-0.5,-0.5 - \y * 2);
				\draw[black,very thick] (0.5, 0.5 - \y * 2) -- (0.5,-0.5 - \y * 2);
				\draw[black,very thick] (1.5, 0.5 - \y * 2) -- (1.5,-0.5 - \y * 2);}
			\node at (-2,0){5};
			\node at (-1,0){8};
			\node at (0,0){3};
			\node at (1,0){9};
			\node at (2,0){1};

			\node at (-2,-2){1};
			\node at (-1,-2){8};
			\node at (0,-2){3};
			\node at (1,-2){9};
			\node at (2,-2){5};
			
			\node at (-2,-4){1};
			\node at (-1,-4){3};
			\node at (0,-4){8};
			\node at (1,-4){9};
			\node at (2,-4){5};
			
			\node at (-2,-6){1};
			\node at (-1,-6){3};
			\node at (0,-6){5};
			\node at (1,-6){9};
			\node at (2,-6){8};
			
			\node at (-2,-8){1};
			\node at (-1,-8){3};
			\node at (0,-8){5};
			\node at (1,-8){8};
			\node at (2,-8){9};

			\draw[red,very thick, <->] (-2,0.5) to[out=60, in=120] (2,0.5);
			\draw[red,very thick, <->] (-1,-1.5) to[out=60, in=120] (0,-1.5);
			\draw[red,very thick, <->] (0,-3.5) to[out=60, in=120] (2,-3.5);
			\draw[red,very thick, <->] (1,-5.5) to[out=60, in=120] (2,-5.5);
			\end{tikzpicture}
		\end{center}
	\end{minipage}

\section{Divide \& Conquer Sortierverfahren}

\section{Heap Sortierverfahren}

\section{Binäre Suchbäume}

\section{AVL-Bäume}

\section{Hashing und Hashtabellen}

\end{document}